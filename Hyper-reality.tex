\documentclass[10pt, oneside]{article}
\usepackage{amsmath, amsthm, amssymb, calrsfs, wasysym, verbatim, bbm, color, graphics, geometry, cite}
\geometry{tmargin=.75in, bmargin=.75in, lmargin=.75in, rmargin = .75in} 


\newcommand{\R}{\mathbb{R}}
\newcommand{\C}{\mathbb{C}}
\newcommand{\Z}{\mathbb{Z}}
\newcommand{\N}{\mathbb{N}}
\newcommand{\Q}{\mathbb{Q}}
\newcommand{\Cdot}{\boldsymbol{\cdot}}
\newcommand{\I}{\mathbb{I}}
\newcommand{\HR}{\mathbb{HR}}
\newcommand{\HRI}{\mathbb{HRI}}
\newcommand{\K}{\mathbb{K}}


\newtheorem{thm}{Theorem}
\newtheorem{defn}{Definition}
\newtheorem{conv}{Convention}
\newtheorem{rem}{Remark}
\newtheorem{lem}{Lemma}
\newtheorem{cor}{Corollary}
\newtheorem{example}{Example}
\newtheorem{exe}{Exercise}
\newtheorem{conjecture}{Conjecture}
\newtheorem{keywords}{Keywords}
\title{Hyper-Real and Infinitesimal Analysis}
\author{[Drew Remmenga drewremmenga@gmail.com unaffiliated]}


\begin{document}


\maketitle
\begin{abstract}
   Something about non-standard analysis is bothersome. Namely the deletion of epsilons when they were such a valuable tool. In this work we build on nonstandard analysis and generalize the dual numbers.
\end{abstract}
\begin{keywords}
   hyper-complex numbers, nonstandard analysis, rings
\end{keywords}
\section{Basic Properties of Infinitesimals}
\textbf{Definition 1 (Hyper-real ring).} Let $\K$ be a field (e.g., $\mathbb{R}$ or $\mathbb{C}$). The \emph{hyper-real ring} $\HR[\K]$ is the commutative ring of formal Laurent polynomials in an indeterminate $\epsilon$ with coefficients in $\K$:
\[
\HR[\K] = \left\{ \sum_{n=k}^{m} a_n \epsilon^n  \;\middle|\; a_n \in \K,\; k,m \in \mathbb{Z},\; k \leq m \right\}
\]
where $\epsilon$ satisfies $\epsilon^n \epsilon^m = \epsilon^{n+m}$ for all $n,m \in \mathbb{Z}$ \cite{robinson1966}. The operations are:
and where $\epsilon$ satisfies $\lim_{x \to 0} x$ and $\epsilon \neq 0$.
\begin{align*}
\text{Addition: } & \sum a_n \epsilon^n + \sum b_n \epsilon^n = \sum (a_n + b_n) \epsilon^n \\
\text{Multiplication: } & \left( \sum a_i \epsilon^i \right) \left( \sum b_j \epsilon^j \right) = \sum_{n} \left( \sum_{i+j=n} a_i b_j \right) \epsilon^n
\end{align*}


\subsection*{Proof of Ring Axioms for $\HR[\K]$}


\begin{proof}
We verify all ring axioms for $\HR[\K]$:


\textbf{Additive Closure:} For $v = \sum_{n=k_v}^{m_v} a_n \epsilon^n$, $w = \sum_{n=k_w}^{m_w} b_n \epsilon^n$, let $k = \min(k_v,k_w)$, $m = \max(m_v,m_w)$. Then:
\[
v + w = \sum_{n=k}^{m} (a_n' + b_n') \epsilon^n \in \HR[\K]
\]
where $a_n' = a_n$ if $k_v \leq n \leq m_v$ (0 otherwise), and similarly for $b_n'$.


\textbf{Additive Associativity:} For $u,v,w \in \HR[\K]$:
\[
(u + v) + w = \sum [(a_n + b_n) + c_n] \epsilon^n = \sum [a_n + (b_n + c_n)] \epsilon^n = u + (v + w)
\]
by associativity in $\K$.


\textbf{Additive Identity:} $0 = \sum 0 \cdot \epsilon^n$ satisfies $v + 0 = v$ for all $v$.


\textbf{Additive Inverses:} For $v = \sum a_n \epsilon^n$, define $-v = \sum (-a_n) \epsilon^n$. Then $v + (-v) = 0$.


\textbf{Commutative Addition:} $v + w = \sum (a_n + b_n) \epsilon^n = \sum (b_n + a_n) \epsilon^n = w + v$.


\textbf{Multiplicative Closure:} For $v = \sum_{i=k_v}^{m_v} a_i \epsilon^i$, $w = \sum_{j=k_w}^{m_w} b_j \epsilon^j$:
\[
v \cdot w = \sum_{n=k_v+k_w}^{m_v+m_w} \left( \sum_{\substack{i+j=n \\ k_v \leq i \leq m_v \\ k_w \leq j \leq m_w}} a_i b_j \right) \epsilon^n \in \HR[\K]
\]
The inner sum has finitely many terms since $i,j$ are bounded.


\textbf{Multiplicative Associativity:} Follows from associativity in $\K$ and $\epsilon^{i}( \epsilon^{j} \epsilon^{k}) = \epsilon^{i+j+k} = (\epsilon^{i} \epsilon^{j})\epsilon^{k}$.


\textbf{Distributivity:} For $u,v,w \in \HR[\K]$:
\begin{align*}
u(v + w) &= \sum_i a_i \epsilon^i \sum_j (b_j + c_j) \epsilon^j \\
&= \sum_n \sum_{i+j=n} a_i (b_j + c_j) \epsilon^n \\
&= \sum_n \left( \sum_{i+j=n} a_i b_j + \sum_{i+j=n} a_i c_j \right) \epsilon^n \\
&= uv + uw
\end{align*}


\textbf{Multiplicative Identity:} $1 = 1\epsilon^0$ satisfies $1 \cdot v = v$ for all $v$.


Thus $\HR[\K]$ is a commutative ring.
\end{proof}
\subsection*{Derivative Properties}
Derivatives in this space are different than in $\C$ or in $\R$. To show this we shall use the limit definition of the derivative on a simple function: $f(x) = x^2$.
\begin{proof}
\begin{align*}
   f'(x) &= \lim_{h \to 0} \frac{f(x+h)-f(x)}{h} \\
   &= \lim_{h \to 0} \frac{(x+h)^2 - x^2}{h} \\
   &= \lim_{h \to 0} \frac{x^2 + 2xh + h^2 - x^2}{h} \\
   &= \lim_{h \to 0} 2x + h \\
   &= 2x + \epsilon
\end{align*}
\end{proof}
There is no non-trivial function with its own derivative in this space.
\begin{proof}
   We take:
\begin{align*}
   f(x) &= f'(x) \\
   f(x) &= \lim_{h \to 0} \frac{f(x+h)-f(x)}{h} \\
   f(x)(\epsilon + 1) &= f(x+\epsilon)\\
   f(x) &= 0
\end{align*}
\end{proof}
\subsection*{Integral Properties}
Integration in this space is different than in $\C$ or $\R$ as well. To show this we shall compute the antiderivative of a trivial function. Namely $f(x) = x$.
\begin{proof}
\begin{align*}
   f'(x) &= x \\
   f(x) &= \frac{x^2}{2} - \frac{\epsilon x}{2}
\end{align*}
\end{proof}
There is no non-trivial function that is its own antiderivative in this space besides the zero function.
\begin{proof}
   See the proof there is no $f(x) = f'(x)$.
\end{proof}
\subsection*{Recovering Hypercomplex Numbers}
If we truncate a power of $\epsilon$ such that $\epsilon^k = \{-1,0,+1\}$ for some $k \in \N$ we recover the hypercomplex numbers.
\begin{proof}
   A hypercomplex number is an element of a unital algebra generated by a basis in $\{1, i_1, i_2, ..., i_n\}$ with coefficients in a field \cite{Taber1904} such that $i^2_k = \{-1,0,+1 \}$.
   We can map directly powers of $\epsilon^n$ to these $i_n$ components and then take $n^2 = k$ for $n,k \in \N$ and we are done.
\end{proof}
\bibliographystyle{plain}  % or another style like alpha, unsrt, etc.
\bibliography{references.bib}  % the name of the .bib file
\end{document}







